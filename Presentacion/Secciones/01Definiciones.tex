%%%%%%%%%%%%%%%%%%%%%%%%%%%%%%%%%%%%%%%%%%%%%%%%%%%%%%%%
\fondo{celeste}
\section{Conceptos básicos}
\fondo{blanco}
%%%%%%%%%%%%%%%%%%%%%%%%%%%%%%%%%%%%%%%%%%%%%%%%%%%%%%%%

%%%%%%%%%%%%%%%%%%%%%%%%%%%%%%%%%%%%%%%%%%%%%%%%%%%%%%%%
\begin{frame}
    \frametitle{Definición de Procesamiento y Análisis de Imágenes}
    \begin{columns}
    \column{.5\textwidth}
    \begin{block}{Procesamiento de Imágenes:}
        \begin{itemize}
            \item Ingresa una imagen y sale una imagen.
            \item Ejemplo: Segmentación.
        \end{itemize}
    \end{block}
    
    \column{.5\textwidth}
    \begin{block}{Análisis de Imágenes:}
        \begin{itemize}
            \item Ingresa una imagen y sale un resultado.
            \item Ejemplo: Clasificación.
        \end{itemize}
    \end{block}
    \end{columns}
\end{frame}
%%%%%%%%%%%%%%%%%%%%%%%%%%%%%%%%%%%%%%%%%%%%%%%%%%%%%%%%

%%%%%%%%%%%%%%%%%%%%%%%%%%%%%%%%%%%%%%%%%%%%%%%%%%%%%%%%
\begin{frame}
    \tikzstyle{block} = [rectangle, draw, rounded corners]

\begin{center}
    
\begin{tikzpicture}[node distance=1.5cm, auto]
    % Nodos
    \node (camara) [node distance=0.5cm] 
        {\scalebox{2.5}{\faCamera}};
    \node (mundo) [node distance=1cm, right of=camara] 
        {\rotatebox{90}{\textbf{Mundo circundante}}};
    \node (captura) [block, right of=mundo, fill=celeste!60] 
        {\rotatebox{90}{Captura de la imagen}};
    \node (pretratamiento) [block, right of=captura, fill=celeste!50] 
        {\rotatebox{90}{Pre-tratamiento}};
    \node (segmentacion) [block, right of=pretratamiento, fill=celeste!40] 
        {\rotatebox{90}{Segmentación}};
    \node (extraccion) [block, right of=segmentacion, node distance=1.8cm, fill=celeste!30] 
        {\rotatebox{90}{\parbox{3cm}{\centering Extracción de caracterísiticas}}};
    \node (deteccion) [block, right of=extraccion, node distance=2.3cm, fill=celeste!20] 
        {\rotatebox{90}{\parbox{3.5cm}{\centering Detección \\ Reconocimiento \\ Interpretación}}};
    \node (resultado) [right of=deteccion, node distance=2cm] 
        {\rotatebox{90}{\textbf{Resultado}}};
    \node (result) [right of=resultado, node distance=1cm] 
        {\scalebox{2.5}{\faCheckSquare}};

    % Flechas
    \draw[->] (mundo) -- (captura);
    \draw[->] (captura) -- (pretratamiento);
    \draw[->] (pretratamiento) -- (segmentacion);
    \draw[->] (segmentacion) -- (extraccion);
    \draw[->] (extraccion) -- (deteccion);
    \draw[->] (deteccion) -- (resultado);
\end{tikzpicture}
\end{center}

\end{frame}
